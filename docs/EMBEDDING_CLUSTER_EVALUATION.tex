\documentclass[11pt]{article}
\usepackage[T1]{fontenc}
\usepackage[utf8]{inputenc}
\usepackage{geometry}
\geometry{margin=1in}
\usepackage{amsmath,amssymb}
\usepackage{booktabs}
\usepackage{mathtools}
\usepackage{natbib}
\usepackage{hyperref}

\newcommand{\R}{\mathbb{R}}
\newcommand{\norm}[1]{\left\lVert #1 \right\rVert}
\newcommand{\E}{\mathbb{E}}

\title{Assessing Clustering Quality in Patient-Specific ECG Embeddings}
\author{}
\date{\today}

\begin{document}
\maketitle

\begin{abstract}
High-quality patient-specific embeddings must simultaneously maintain compact intra-patient cohesion and well-separated inter-patient structure. This review summarizes the most informative validation criteria for clustering structure in learned embedding spaces, emphasizing indices grounded in intra-cluster scatter and inter-cluster separation. We provide the mathematical formulation of each metric, discuss their statistical underpinnings, and outline practical guidance tailored to electrocardiogram (ECG) representation learning.
\end{abstract}

\section{Introduction}
Let $\mathcal{X} = \{x_i\}_{i=1}^N \subset \R^d$ denote embedding vectors extracted from ECG exams and let $\mathcal{C} = \{C_k\}_{k=1}^K$ denote the partition induced either by patient identifiers or a clustering algorithm. Robust evaluation of clustering quality is central for downstream biometric verification, anomaly detection, and longitudinal monitoring in cardiology. Indices that combine intra-cluster cohesion with inter-cluster separation are especially relevant because they align with the clinical expectation that repeated exams from the same patient should form tight clusters while different patients remain distinguishable.

This note formalizes leading validity indices, categorizes them by their reliance on scatter or density statistics, and highlights implementation details that preserve numerical stability in high-dimensional ECG embeddings.

\section{Notation and Basic Quantities}
For each cluster $C_k$, define its cardinality $n_k = |C_k|$ and centroid
\begin{equation}
  \mu_k = \frac{1}{n_k} \sum_{x_i \in C_k} x_i.
\end{equation}
The global centroid is given by $\bar{x} = \frac{1}{N} \sum_{i=1}^{N} x_i$.
We use two generic building blocks:
\begin{align}
  S_k^{(p)} &= \left( \frac{1}{n_k} \sum_{x_i \in C_k} \norm{x_i - \mu_k}^p \right)^{1/p} \quad \text{(within-cluster dispersion)}, \label{eq:scatter} \\
  M_{kl} &= \norm{\mu_k - \mu_l} \quad \text{(centroid separation)}. \label{eq:separation}
\end{align}
When $p=2$, $S_k^{(2)}$ coincides with the root mean square deviation. Alternative scatter measures include the maximum intra-cluster distance (diameter)
\begin{equation}
  \Delta(C_k) = \max_{x_i, x_j \in C_k} \norm{x_i - x_j},
\end{equation}
while an inter-cluster distance can be defined via single, complete, or average linkage; we denote a generic choice by $\delta(C_k, C_l)$.

\section{Scatter-Separation Indices}
\subsection{Between-Within Ratio}
A simple but informative baseline is the between-within ratio
\begin{equation}
  \operatorname{BWR} = \frac{\frac{2}{K(K-1)} \sum_{k < l} M_{kl}}{\frac{1}{K} \sum_{k=1}^{K} S_k^{(2)}},
\end{equation}
where larger values imply better separation relative to cohesion. Although scale-dependent, BWR is a useful diagnostic when monitored across training epochs.

\subsection{Calinski--Harabasz Index}
The Calinski--Harabasz (CH) index \citep{calinski1974dendrite} compares between-cluster scatter $B$ and within-cluster scatter $W$:
\begin{align}
  W &= \sum_{k=1}^{K} \sum_{x_i \in C_k} (x_i - \mu_k)(x_i - \mu_k)^{\top}, \\
  B &= \sum_{k=1}^{K} n_k (\mu_k - \bar{x})(\mu_k - \bar{x})^{\top}, \\
  \operatorname{CH} &= \frac{\operatorname{tr}(B)/(K-1)}{\operatorname{tr}(W)/(N-K)}.
\end{align}
Higher CH values indicate more distinct and compact clusters. The ratio is invariant under affine scaling, making it suitable for comparing embeddings with different variance levels.

\section{Pairwise Distance Indices}
\subsection{Silhouette Coefficient}
The silhouette coefficient \citep{rousseeuw1987silhouettes} evaluates each embedding individually. Define the intra-cluster distance for a point $x_i \in C_{k(i)}$ as
\begin{equation}
  a_i = \frac{1}{n_{k(i)} - 1} \sum_{\substack{x_j \in C_{k(i)} \\ j \neq i}} \norm{x_i - x_j},
\end{equation}
and the lowest mean distance to points in another cluster as
\begin{equation}
  b_i = \min_{l \neq k(i)} \frac{1}{n_l} \sum_{x_j \in C_l} \norm{x_i - x_j}.
\end{equation}
The silhouette score is then
\begin{equation}
  s_i = \frac{b_i - a_i}{\max\{a_i, b_i\}}, \qquad S = \frac{1}{N} \sum_{i=1}^{N} s_i.
\end{equation}
Values near $1$ imply strong clustering, values around $0$ indicate overlap, and negative values signal possible misassignment.

\subsection{Dunn Index}
The Dunn index \citep{dunn1973separability} focuses on extremal distances:
\begin{equation}
  D = \frac{\min_{i \neq j} \delta(C_i, C_j)}{\max_{k} \Delta(C_k)}.
\end{equation}
A large Dunn index combines large inter-cluster separation with small intra-cluster diameter. Robust computation benefits from using average-linkage for $\delta$ to mitigate noise sensitivity in ECG embeddings.

\section{Cluster-Centric Aggregation Indices}
\subsection{Davies--Bouldin Index}
The Davies--Bouldin (DB) index \citep{davies1979cluster} penalizes clusters whose scatter is large relative to their separation:
\begin{equation}
  R_{kl} = \frac{S_k^{(2)} + S_l^{(2)}}{M_{kl}}, \qquad \operatorname{DB} = \frac{1}{K} \sum_{k=1}^{K} \max_{l \neq k} R_{kl}.
\end{equation}
Lower DB values indicate better clustering. Because DB relies on centroids, it is computationally efficient for high-cardinality ECG cohorts.

\subsection{Xie--Beni Index}
The Xie--Beni (XB) index \citep{xie1991validity} was introduced for fuzzy clustering but applies to hard partitions as well:
\begin{equation}
  \operatorname{XB} = \frac{\sum_{k=1}^{K} \sum_{x_i \in C_k} \norm{x_i - \mu_k}^2}{N \cdot \min_{i \neq j} M_{ij}^2}.
\end{equation}
Smaller XB values reflect compact clusters widely separated by their centroids. XB is sensitive to outliers; trimming the top percentile of intra-cluster distances improves robustness in ECG applications.

\subsection{SD and SD$_{\text{BW}}$ Indices}
Halkidi et al. propose the SD and SD$_{\text{BW}}$ indices \citep{halkidi2001clustering} that combine scatter with density uniformity. SD balances average scattering $\bar{S} = \frac{1}{K} \sum_k S_k^{(2)}$ against the standard deviation of inter-centroid distances, while SD$_{\text{BW}}$ augments DB with a density term computed over neighborhood radii. These indices detect over-partitioning, a common failure mode when embeddings collapse locally.

\section{Temporal and Patient-Level Diagnostics for ECG}
ECG embeddings often include repeated exams per patient. We recommend complementing static indices with longitudinal diagnostics:
\begin{enumerate}
  \item \textbf{Trajectory dispersion}: For patient $p$, compute $\sigma_p^2 = \frac{1}{n_p} \sum_{x_i \in C_p} \norm{x_i - \mu_p}^2$ and monitor its variance across patients.
  \item \textbf{Intra/inter ratio}: Define $\rho = \frac{\frac{1}{P} \sum_{p} \sigma_p}{\min_{p \neq q} \norm{\mu_p - \mu_q}}$; values $\rho < 0.5$ indicate strong patient signatures \citep{biel2001ecg}.
  \item \textbf{Epochal drift}: Evaluate CH or DB on validation splits after each training epoch to detect collapse early.
\end{enumerate}

\section{Practical Recommendations}
\begin{itemize}
  \item \textbf{Use complementary indices}: Combine CH (global variance ratio), silhouette (pointwise reliability), and DB (worst-case cluster overlap) to capture different failure modes.
  \item \textbf{Normalize embeddings}: Enforce $\ell_2$ normalization before distance computations to make indices comparable across training runs.
  \item \textbf{Mitigate class-size imbalance}: Weight cluster contributions by $n_k$ when computing scatter to avoid dominance by prolific patients.
  \item \textbf{Bootstrap uncertainty}: Estimate confidence intervals for each index via patient-wise resampling to quantify statistical significance.
  \item \textbf{Align with clinical metadata}: Stratify indices by rhythm class or acquisition device to distinguish modeling errors from data heterogeneity.
\end{itemize}

\section{Conclusion}
Distance-based clustering indices remain indispensable for validating ECG embeddings. Metrics such as Calinski--Harabasz, silhouette, Davies--Bouldin, Dunn, and Xie--Beni offer complementary perspectives on intra-cluster cohesion and inter-cluster separation. Proper normalization, resampling, and longitudinal monitoring ensure that patient-specific structure is preserved without suffering embedding collapse.

\bibliographystyle{plainnat}
\bibliography{embedding_cluster_evaluation}

\end{document}
